\chapter{Experiment Design and Results}
\paragraph{Overview} This chapter describes the experiment that we conducted based on the system design showed in chapter~\ref{sec} and our final results, consisting in the trained model and its evaluation. In particular, in section~\ref{sec:nn} we present the selected architecture among the different possibilities in recent literature. Section~\ref{sec:input_metrics_sample} describes all the necessary details about the input selection, the chosen evaluation metrics and how results are produced. Section~\ref{sec:results} presents the resulting model, the metrics associated to it and some examples of generated levels. 
\section{Neural Network Architecture and Training Algorithm}
\label{sec:nn}
\subsection{Network Architecture and Hyper-parameters}
\label{sec:networkarch}

\paragraph{Chosen Architecture}
 
\paragraph{Advantages} %TODO Parla dei vantaggi e soprattutto del problema della (validation) loss che non sempre è sinonimo di qualità dei samples, citando i paper.
\label{sec:arch_advantages}

\subsection{Techniques and "GAN Tricks" used}
\subsection{Training Algorithm}
% Show the table with the steps at each iteration and motivate them


\section{Input selection, Metrics and Sample Method}
\label{sec:input_metrics_sample}
\subsection{Input data}
 % TODO Inserire Constraints su tfrecords
\label{sec:InputSelection}

\subsection{Evaluation metrics}
%TODO: Parla del fatto che di solito si usa inception score o simili. Cita anche il paper in cui proponevano modi alternativi che però dipendevano comunque da reti neurali e i risultati ottenuti non sono stati soddisfacenti. Poi proponi e spiega le metriche e parla del fatto che è comunque difficile trovare metriche sensate per comparare il tipo di dati su cui lavoriamo.
\label{sec:evaluation}
\subsection{Sampling the network}


\section{Results}
\label{sec:results}
\subsection{Resulting Model}
\label{sec:sampling}
% Alcune considerazioni su come viene campionata la rete 
% Eventuale sezione sull'interpolazione di livelli
\section{Generated Samples}
\section{In-Game Demonstration}
\section{Summary}