\newglossaryentry{WAD}
{
  name=WAD,
  description={Default format of package files for the DOOM / DOOM II video-games. "WAD" is an acronym for "Where's all the Data?" \cite{wadmeaning} },  plural=WADs
}
\newglossaryentry{lump}
{
	name=Lump,
	description={ Any section of data within a \gls{WAD} file. }
}
\newglossaryentry{thing}
{
	name=Thing,
	description={ Any deployable asset of a DOOM level, such as Enemies, Power-Ups, Weapons, Decorations, Spawners, Etc. }
	plural=Things
}
\newglossaryentry{sector}
{
	name=Sector,
	description={A Sector in a DOOM level is any closed area (with possibly invisible walls) that has a constant floor and ceiling height and texture.  }
	plural=Things
}
\newglossaryentry{deathmatch}
{
	name=Deathmatch,
	description={ Game mode in which two or more players compete against each other in achieving the highest number of kills before a timeout or a player reaches a predetermined score. }
}

\newglossaryentry{MU}{
	name=MU,
	description={Map Units, Coordinate unit used in Doom Rendering Engine. }
}

\newglossaryentry{DU}{
	name=DU,
	description={Doom Units, see \gls{MU} }
}
