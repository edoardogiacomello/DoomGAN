\chapter{Introduction}

Content creation in video-games is one of the most time consuming and difficult tasks in the process of developing a good quality product and producing interesting content often requires design experts. Content of a video-game belongs to two categories: Functional content is related to the game mechanics, in contrast to Non-Functional content, which usually serves a cosmetic purpose or has marginal effect on the game from a point of view of the player actions. In this work we only consider the problem of level design, which belongs to the first category. \\*
Level design is an essential part in the development of many video-game genres such as Platform Games and First Person Shooters; in many cases a good level design contributed to the enormous success of many video-games. \\*
Besides the costs of level design, other problems arose in the early history of video games: often the memory resources were scarce and the content couldn't be stored in memory. \textit{Procedural content generation} came up as a solution to solve this issue by generating levels and other content by means of an algorithm rather then an human designer.

Many early games used Procedural Content Generation for overcoming the memory limitations on the machines that were used to play \cite{pcgbook}. Notable examples in the early history are \textit{Elite}\cite{game:elite}, a space simulation in which procedural generation is used to create the game universe and \textit{Rogue}\cite{game:rogue}, a dungeon-crawling game in which dungeon rooms and the hallways are generated by means of an algorithm. \\*
With the increase of computing capabilities over time, the problem of storage became less severe. Nonetheless, PCG remained as a feature in many video-games, often playing a central role in the design of many games. For example in \textit{Diablo}\cite{game:diablo} every map and item is procedurally generated and many other games utilize software for automatizing some processes that would be extremely expensive if done manually, like populating an area with vegetation \cite{pcgpaper}.\\*
Recently, PCG have been often applied to increase \textit{re-playability}: if a game is played many time, the game experience could be always different. An example is given by \textit{Minecraft}\cite{game:minecraft}, in which an initial part of the world is procedurally generated at the start of the game and it is expanded basing on the world seed as the player explores new areas. Other titles that make use of procedurally generated content as a fundamental design tool are \textit{Dwarf Fortress}\cite{game:dwarf}, \textit{Elite: Dangerous}\cite{game:elitedangerous} and \textit{No Man's Sky}\cite{game:nomanssky}, only to cite some of them.


Thanks to the increasing interest that machine learning topics gained in recent times, it is possible to apply new methodologies to the problem of content generation. In particular, \citeauthor{PCGML} propose a new practice called PCGML (Procedural Content Generation via Machine Learning) \cite{PCGML} as the generation of game content by machine learning models that have been trained on existing content. This type of approach differs from "classical" procedural content generation because it does not imply a search in the content space, but the model directly generates the content.  \citeauthor{PCGML} propose in their article a survey on the work that has been already done in the field and present many possible applications of PCGML.


\section{Scope}
\label{sec:scope}
In this work we study the applicability of Generative Adversarial Networks to the problem of generating new maps for the First Person Shooter game DOOM in the context of Procedural Content Generation via Machine Learning. We propose an alternative model to the classical \textit{procedural content generation}, inheriting the advantages introduced by PCGML. This allows creation of new levels without the need of having an human expert to embed their knowledge during the process, but exploiting patterns in training data instead. Our work takes place in the domain of first person shooter maps for which few work has been done yet using this type of models, since they differ from the commonly used platform maps which exhibit a sequential structure and a linear traversal. To assess the usability of this model in a real generation environment we study the advantages of adding input parameters in the form of level features, which allows to customize the generated levels. For studying the capabilities of the network models we produced we design three experiments that shows how the presence of input features affects level generation. 

\section{Related Work}
\label{sec:relatedwork}
\subsection{PCGML in Video-Games} 
\citeauthor{PCGML} show that a good amount of work has been done with different use cases, methods and data representation \cite{PCGML}. However, the domain in which the majority of work related to level design is done is the one of platform games. For example, \citeauthor{mariongrams} \cite{mariongrams} uses \textit{n-grams} to generate new levels for \textit{Super Mario}\cite{game:supermario}, \citeauthor{levelsautoencoder} use \textit{autoencoders}, while \citeauthor{mariomarkovchains} experiment an approach based on Markov Chains \cite{mariomarkovchains}. 
An exception to this type of domain that is still related to level generation is the work of \citeauthor{zeldalevels} in \cite{zeldalevels} where the authors present a method for generating levels of \textit{Legend of Zelda}\cite{game:zelda} using Bayes Nets for the high level topological structure and Principal Component Analysis for generating the rooms. Our work proposes instead a method for generating the whole level topology using a single model, with the possibility of easily adding more features or eventually applying the same structure to another dataset. \newline
In their work, \citeauthor{resourcelocation} \cite{resourcelocation} use neural networks for predicting resources location in \textit{StarCraft II}\cite{game:starcraft} maps. Although the data domain is similar to the one used in our work, the problem only focused on resource placement rather then map topology generation and requires the image of an already existing level as input. Moreover, we make use of Generative Adversarial Networks, which is a particular setting in which a generator is able to produce new samples using a vector of noise as input. \\* 
In the settings of Generative Adversarial Networks, \citeauthor{heightgen} propose a method for generating realistic height level maps for video-games \cite{heightgen}, which is more applicable to realistic landscapes rather than fictional indoor environments such those of First Person Shooters. This kind of model have also been applied to non-functional content generation in the work of \citeauthor{spritegen}, in which GANs are used to generate 2d sprites \cite{spritegen}.
 

\subsection{Video Game Level Corpus}
One of the problems with this type of generative models, as explained in \cite{PCGML} is that they require a large amount of data to be optimized. Unfortunately, the domain of video-games levels does not benefit of large datasets to work with, and generally levels from different video games does not share common data structures. \citeauthor{VGLC} created the \textit{Video Game Level Corpus (VGLC)}, a collection of game levels represented in multiple formats. Using this format as starting point, we collected about 9000 user-generated Doom levels of different size and designed an extended representation that better fit our needs, while still keeping the dataset compatible with the original VGLC representation. Although VGLC provides the parser they used for data generation, we wrote a new parser which better integrates with our system and feature representation, and can also be used as a stand-alone parser for future researches.

\subsection{Generative Adversarial Networks}
Generative Adversarial Networks are a recent generative model based on Artificial Neural Networks. This type of model allows to learn the data distribution of a dataset and generate synthetic data that exhibit similar characteristics to the real data. Among all the domains in which GANs have been already applied, that of images is one of the most prominent. For example, generation tasks are commonly applied to the handwritten digits (MNIST \cite{dataset:MNIST}) dataset, human faces (CelebA \cite{dataset:celebA}) and bedrooms (LSUN \cite{dataset:LSUN}) as in \cite{gan:dcgan}, but a large amount of creative work is done with many other datasets such as birds, flower \cite{gan:birds}, and other type of images. Another task which involves pictures is image-to-image translation: \citeauthor{image-to-image} investigates GANs as a general solution to this problem in several settings \cite{image-to-image} such as image colourisation, segmented image to realistic scene conversion or  the generation of realistic objects starting from hand-drawn input from the user (Figure~\ref{fig:img-to-img}). The GAN approach has been also used in many other domains such as frame prediction in videos \cite{gan:frameprediction} and sound generation \cite{gan:sound}, being a research area in rapid expansion.

\begin{figure}[h!]
	\begin{center}
		\includegraphics[width=1\linewidth]{pix2pix}
	\end{center}
	
	\captionsetup{width=1\linewidth}
	\caption[Pix2Pix example from \citeauthor{image-to-image}]{Examples from the work of \citeauthor{image-to-image} on image-to-image translation problem \cite{image-to-image}.}
	\label{fig:img-to-img}
	\medskip
	
\end{figure}



\section{Thesis Structure}
\paragraph{} Chapter~\ref{ch:theory} describes the theory which is needed to understand how we designed our system and motivates the choice of DOOM as the game we used in this work. Chapter~\ref{ch:dataset} describes in detail the data we used in our work and how we converted it in order to make it functional for our system. Chapter \ref{ch:system_design} first describes the system from an high level perspective defining the possible use cases and processes, then detail the neural network model we designed for our system and describes some additional metrics we designed to monitor the training process. Chapter~\ref{ch:experiment} shows what choices are to make for designing an experiment using our system, then defines three experiments which let us study the trained models. Chapter~\ref{sec:results} reports and discuss the most relevant results obtained during the training process and by running the experiments. Chapter~\ref{ch:conclusions} shows our general considerations about our work, while highlighting the open problems and proposing further work on this topic.

\section{Summary}
In this chapter we introduced the problem of Level Design in video-game industry and the solutions that have been historically applied for approaching the problem. Then we referenced a new area of research in which this work take place and defined its scope.
We also described the main differences between our research domain and the domains of the main contributions in each research area we considered, highlighting the differences between our work and other contributions. In the next chapter we give a more in-depth description of the Generative Adversarial Network models and considerations about the choice of the game. 
