\chapter{Introduction}
%TODO: Some description

\section{Background: Level Design}
%% Parla dei principali problemi del level design, di quanto sia laborioso per un designer creare livelli interessanti
\section{Related Work}
\subsection{Procedurally Generated Content}
% Parla della storia del PGC e di come si sia proposto come soluzione al problema del design
\subsection{Procedural Content Generation via Machine Learning (PCGML)}
% What is proposed from snodgrass at al
\subsection{Video Game Level Corpus}
% Parla di VGLC E DEL LORO PARSER e del nuovo parser

\subsection{Generative Adversarial Networks}
% Parla di dei risultati raggiunti dalle gan e di come però lavorino solo su immagini vere

\section{Scope}
% Si vuole proporre un modello alternativo che in linea teorica dovrebbe portare alcuni vantaggi rispetto al PCG, ovvero un approccio basato su machine learning che noi applichiamo  a DOOM, che permette di aiutare il designer non solo nel processo di generazione dei livelli ma anche nel sfruttare caratteristiche intriseche nei dati senza una esplicita definizione delle stesse da parte di un esperto (questo al costo di una certa quantità di dati da analizzare). Si vuole descrivere un sistema (...) ed evidenziare le principali problematiche incontrate e le motivazioni per cui questi problemi non sono affrontabili con le tecniche in uso correntementente.
% Cita anche le differenze del nostro tipo di dati rispetto a quello che è comunemente usato
\section{Thesis Structure}
%TODO: Finish this
\section{Summary}